\documentclass[12pt,a4paper]{scrartcl}

\author{Sebastian Hirnschall, Rafael Dorigo}
%% (C) Hirnschall Sebastian, Rafael 2019
\date{\today}


\usepackage[
backend=biber,
style=authoryear-icomp,    % Zitierstil
isbn=false,                % ISBN nicht anzeigen, gleiches geht mit nahezu allen anderen Feldern
pagetracker=true,          % ebd. bei wiederholten Angaben (false=ausgeschaltet, page=Seite, spread=Doppelseite, true=automatisch)
maxbibnames=50,            % maximale Namen, die im Literaturverzeichnis angezeigt werden (ich wollte alle)
maxcitenames=3,            % maximale Namen, die im Text angezeigt werden, ab 4 wird u.a. nach den ersten Autor angezeigt
autocite=inline,           % regelt Aussehen für \autocite (inline=\parancite)
block=space,               % kleiner horizontaler Platz zwischen den Feldern
backref=true,              % Seiten anzeigen, auf denen die Referenz vorkommt
backrefstyle=three+,       % fasst Seiten zusammen, z.B. S. 2f, 6ff, 7-10
date=short                % Datumsformat
]{biblatex}

\addbibresource{./refs.bib}

\usepackage{longtable}
%\usepackage{hyperref}
\usepackage{amsmath}% http://ctan.org/pkg/amsmath
\usepackage[ngerman]{cleveref} %referenzen fur Abbildungen
\usepackage{graphicx}
\usepackage{listings}
\usepackage{esdiff}
\usepackage[utf8]{inputenc}
\usepackage[ngerman]{babel}
\usepackage[T1]{fontenc}
\usepackage{graphicx}
\usepackage{amssymb}
\usepackage{geometry}% http://ctan.org/pkg/geometry
\usepackage{amsthm}
\usepackage{tocloft}
\usepackage{framed}
\usepackage{mathtools}
\usepackage{color}
\usepackage{multirow}
\usepackage{textcomp}
\usepackage{float}
%\usepackage[dvipsnames]{xcolor}
\usepackage{mathtools}
\usepackage{caption}
\usepackage{subcaption}

\usepackage[table,xcdraw,dvipsnames]{xcolor}

\usepackage{fancyvrb}




%\pagestyle{headings}

\setcounter{secnumdepth}{5}
\setcounter{tocdepth}{5}

%\pagestyle{headings}

\usepackage{fancyhdr}
\pagestyle{fancy}
%
\rhead{ }
%\rhead[re]{\textbf{\nouppercase{\leftmark}}}
\chead{}
\lhead{\leftmark}
%%
\lfoot{Sebastian Hirnschall, Rafael Dorigo}
\cfoot{}
\rfoot{\thepage}
%%
\renewcommand{\headrulewidth}{0.2pt}
\renewcommand{\footrulewidth}{0.2pt}
\newcommand{\R}{\mathbb{R}}


\fancypagestyle{general}
{
	\fancyhf{}
	\rhead{}
	%\rhead[re]{\textbf{\nouppercase{\leftmark}}}
	\chead{}
	\lhead{\leftmark}
	\lfoot{Sebastian Hirnschall, Rafael Dorigo}
	\cfoot{}
	\rfoot{\thepage}
}


%listings settings
\definecolor{mygreen}{rgb}{0,0.6,0}
\definecolor{mygray}{rgb}{0.5,0.5,0.5}
\definecolor{mymauve}{rgb}{0.58,0,0.82}
\definecolor{BackgroundGray}{rgb}{0.9,0.9,0.9}

\lstset{ %
	backgroundcolor=\color{BackgroundGray},   % choose the background color; you must add \usepackage{color} or \usepackage{xcolor}
	basicstyle=\footnotesize,        % the size of the fonts that are used for the code
	breakatwhitespace=false,         % sets if automatic breaks should only happen at whitespace
	breaklines=true,                 % sets automatic line breaking
	captionpos=b,                    % sets the caption-position to bottom
	commentstyle=\color{mygreen},    % comment style
	deletekeywords={...},            % if you want to delete keywords from the given language
	escapeinside={\%*}{*)},          % if you want to add LaTeX within your code
	extendedchars=true,              % lets you use non-ASCII characters; for 8-bits encodings only, does not work with UTF-8
	frame=single,	                   % adds a frame around the code
	keepspaces=true,                 % keeps spaces in text, useful for keeping indentation of code (possibly needs columns=flexible)
	keywordstyle=\color{blue},       % keyword style
	language=C,                 	   % the language of the code
	otherkeywords={*,...},           % if you want to add more keywords to the set
	numbers=left,                    % where to put the line-numbers; possible values are (none, left, right)
	numbersep=5pt,                   % how far the line-numbers are from the code
	numberstyle=\tiny\color{mygray}, % the style that is used for the line-numbers
	rulecolor=\color{mygray},         % if not set, the frame-color may be changed on line-breaks within not-black text (e.g. comments (green here))
	showspaces=false,                % show spaces everywhere adding particular underscores; it overrides 'showstringspaces'
	showstringspaces=false,          % underline spaces within strings only
	showtabs=false,                  % show tabs within strings adding particular underscores
	stepnumber=2,                    % the step between two line-numbers. If it's 1, each line will be numbered
	stringstyle=\color{mymauve},     % string literal style
	tabsize=2,	                   % sets default tabsize to 2 spaces
	title=\lstname,                   % show the filename of files included with \lstinputlisting; also try caption instead of title
	emph={int,unsigned,long,vector,char,string},
	emphstyle={\color{ForestGreen}}
}

\Crefname{lstlisting}{Listing}{Listing}


%italic quotes
\newenvironment{italicquotes}
{\begin{quote}\itshape}
	{\end{quote}}


%tableofcontents font
%\renewcommand{\cftchapfont}{\scshape}
\renewcommand{\cftsecfont}{\bfseries}
\addtokomafont{disposition}{\rmfamily}

\newcommand{\spar}{\par\vspace{10pt}\noindent}
\newcommand{\Mod}[1]{\ (\text{mod}\ #1)}


\usepackage{twoopt}
\newcommandtwoopt{\img}[4][0.5cm][0.7]{
	\begin{figure}[!h]
		\vspace{#1}
		\centering
		\includegraphics[width=#2\textwidth]{#3}
		\caption{#4} %\footnotemark}
		\label{fig:#3}
	\end{figure}
	%\footnotetext{#5}
}


\numberwithin{equation}{section} 
%\makeatletter
%\@addtoreset{equation}{section}
%\makeatother


%\newtheorem{theorem}{Theorem}[section]
%\newtheorem{lemma}[theorem]{Lemma}
%\newtheorem{proposition}[theorem]{Proposition}
%\newtheorem{corollary}[theorem]{Corollary}

\newcounter{myalgctr}

\newenvironment{mydef}{%      define a custom environment
	\par\noindent%         create a vertical offset to previous material
	\refstepcounter{myalgctr}% increment the environment's counter
	\textsc{\textbf{Definition} \themyalgctr}% or \textbf, \textit, ...
	\newline
}{\par\bigskip}  %          create a vertical offset to following material
\numberwithin{myalgctr}{section}

\crefname{myalgctr}{Definition}{Definitionen}

%theorem
\newcounter{mytheoremctr}

\newenvironment{mytheorem}{%      define a custom environment
	\refstepcounter{mylemmactr}% increment the environment's counter
	\refstepcounter{mykorollarctr}
	\refstepcounter{mybeispielctr}% increment the environment's counter
	\refstepcounter{mytheoremctr}
	\par \noindent%         create a vertical offset to previous material
	\textsc{\textbf{Satz} \themytheoremctr}% or \textbf, \textit, ...
	\newline\noindent
}{\par\bigskip}  %          create a vertical offset to following material
\numberwithin{mytheoremctr}{subsection}

\crefname{mytheoremctr}{Satz}{Satz}

%korollar
\newcounter{mykorollarctr}

\newenvironment{mykorollar}{%      define a custom environment
	\refstepcounter{mylemmactr}% increment the environment's counter
	\refstepcounter{mytheoremctr}
	\refstepcounter{mybeispielctr}% increment the environment's counter
	\refstepcounter{mykorollarctr}
	\par\noindent%         create a vertical offset to previous material
	\textsc{\textbf{Korollar} \themykorollarctr}% or \textbf, \textit, ...
	\newline\noindent
}{\par\bigskip}  %          create a vertical offset to following material
\numberwithin{mykorollarctr}{subsection}

\crefname{mykorollarctr}{Korollar}{Korollar}

%lemma
\newcounter{mylemmactr}

\newenvironment{mylemma}{%      define a custom environment
	\refstepcounter{mykorollarctr}
	\refstepcounter{mytheoremctr}
	\refstepcounter{mybeispielctr}% increment the environment's counter
	\refstepcounter{mylemmactr}% increment the environment's counter
	\par\noindent%         create a vertical offset to previous material
	\textsc{\textbf{Lemma} \themylemmactr}% or \textbf, \textit, ...
	\newline\noindent
}{\par\bigskip}  %          create a vertical offset to following material
\numberwithin{mylemmactr}{subsection}

\crefname{mylemmactr}{Lemma}{Lemma}

\newcounter{mybeispielctr}

%beispiel
\newenvironment{mybeispiel}{%      define a custom environment
	\refstepcounter{mykorollarctr}
	\refstepcounter{mytheoremctr}
	\refstepcounter{mylemmactr}% increment the environment's counter
	\refstepcounter{mybeispielctr}% increment the environment's counter
	\par\noindent%         create a vertical offset to previous material
	\textsc{\textbf{Beispiel} \themybeispielctr}% or \textbf, \textit, ...
	\newline\noindent
}{\par\bigskip}  %          create a vertical offset to following material
\numberwithin{mybeispielctr}{subsection}

\crefname{mybeispielctr}{Beispiel}{Beispiel}

\newenvironment{myproof}{%      define a custom environment
	\bigskip\noindent%         create a vertical offset to previous material
	\textsc{\textbf{\\Beweis\\}}% or \textbf, \textit, ...
	\indent
}{\qed\par\bigskip}  %          create a vertical offset to following material

\newenvironment{bemerkung}{%      define a custom environment
	\bigskip\noindent%         create a vertical offset to previous material
	\textsc{\textbf{\\Bemerkung.}}% or \textbf, \textit, ...
	\indent
}{\par\bigskip}  %          create a vertical offset to following material

\newcommand{\mpar}[1]{\paragraph*{#1}\mbox{}\par}
\newcommand\norm[1]{\left\lVert#1\right\rVert}
\DeclarePairedDelimiter{\abs}{\lvert}{\rvert}


\pagestyle{fancy}
\fancypagestyle{firststyle}
{
	\fancyhf{}
	\rhead{}
	%\rhead[re]{\textbf{\nouppercase{\leftmark}}}
	\chead{}
	\lhead{\leftmark}
	\lfoot{ Sebastian Hirnschall, Rafael Dorigo}
	\cfoot{}
	\rfoot{\thepage}
}

\crefname{section}{Abschnitt}{Abschnitt}

\begin{document}
	\newgeometry{bottom=1cm,top=1cm,left=1cm,right=1cm}
	\begin{titlepage}
		\begin{flushleft}
				\includegraphics[width=.4\linewidth]{tuwien.png}
		\end{flushleft}	
		\centering
		
		
		\vspace{5cm}
		{\huge\bfseries Numerische Integration\par}
		\vspace{2cm}
		{\Large\itshape Sebastian Hirnschall\\Rafael Dorigo\par}
		\vspace{1cm}
		{\large\itshape Betreut von:\\Markus Wess Dipl.-Ing.\par}
		\vspace{1cm}
		\begin{figure}[!h]
			\vspace{0cm}
			\centering
			\includegraphics[width=.6\linewidth]{./titlepage.png}
		\end{figure}
		
		\vfill
		
		% Bottom of the page
		{\large 24. Dezember 2019\par}
	\end{titlepage}
	\restoregeometry
	
	\thispagestyle{firststyle}
	
	\newpage\noindent
	{\LARGE \bfseries Abstract}
	\newline
	\par\noindent
	Vordergründiges Ziel dieser Arbeit ist es, exponentiell abfallende Funktionen auf dem unbeschränkten Intervall [0,$\infty$) numerisch zu integrieren und verschiedene Vorgehensweisen zu vergleichen. Zunächst werden zwei unterschiedliche M\"oglichkeiten um auch das Integral \"uber unbeschr\"ankte Intervalle zu approximieren vorgestellt und untersucht wie sich verschiedener Eigenschaften der zu integrierenden Funktion auf das Konvergenzverhalten des n\"aherungsverfahrens auswirken. %Um problemlos hohe Geschwindigkeiten zu erreichen, verwenden viele Hashfunktionen Bit-Operatoren
	Anschließend werden die beiden gew\"ahlten Methoden miteinander verglichen.
	Das letzte Kapitel befasst sich mit der Frage, inwiefern sich das Konvergenzverhalten durch die Wahl unterschiedlicher Gewichtsfunktionen \"andert. Dabei wird verglichen wie schnell der Fehler der zuvor gew\"ahlten Vorgehensweisen f\"ur zwei h\"aufig verwendete Gewichtsfunktionen ($e^{-x}$ und $e^{-x^2}$) konvergieren.
	\thispagestyle{firststyle}
	
	\newpage
	\tableofcontents
	\thispagestyle{general}
	\newpage

	\section{Einleitung}
	Um exponentiell abfallende Funktionen auf dem unbeschränkten Intervall [0,$\infty$) zu integrieren, bieten sich zwei Möglichkeiten an. Das Abschneiden des unbeschränkten Intervalls und Anwendung einer Quadraturformel für ein beschränktes Intervall $[0,T]$ für $T > 0$ und die Anwendung einer Gaußquadratur für die Gewichtsfunktion $w(x) = e^{-x}$ (Gauss–Laguerre Quadratur).
	
	\newpage
	\section{Summierte Trapezformel auf einem endlichen Intervall}\label{trapez}
	\begin{mylemma}\label{lemma1.1}
	Sei $f$ : [0,$\infty$) $\rightarrow$ $\R$ eine beschränkte Funktion. Des weiteren definieren wir für eine Menge M $\subset$ [0,$\infty$) das gewichtete Integral von $f$ mit
	\begin{align*}
		Q_M(f) := \int_{M}^{}{f(x)exp(-x)dx}
	\end{align*}
	Das Integral $Q_{[0,\infty)}(f)$ kann man approximieren, indem man für ein T > 0 das Integral auf dem beschränkten Intervall $Q_{[0,T)}(f)$ durch die summierte Trapezformel $Q_{h,T}(f)$ berechnet. Es folgt eine Fehlerabschätzung der Form 
	\begin{align}
		|Q_{[0,\infty)}(f) - Q_{h,T}(f)| \le C_1\varepsilon_T + C_2T\varepsilon_h \label{eq:trapez-fehler}
	\end{align}
	wobei die Terme $\varepsilon_T, \varepsilon_h$ lediglich von $T$ bzw. $h$ abhängen und $C_1, C_2$ jeweils von $T,h$ unabhängig sind. Des weiteren wir vorausgesetzt, dass $f$ hinreichend glatt ist und die Supremumsnorm der zweiten Ableitung von $f$ beschränkt ist.
	\end{mylemma}
	\begin{myproof}
		Mit \autocite[vgl.][40-41]{skript}: \\
		und $C_1 := \norm{f}_{\infty}$ , \quad $\varepsilon_T := e^{-T}$ ,
		\quad $C_2 := \norm{f^{''}}_{\infty}$ , \quad $\varepsilon_h := \frac{h^{2}}{12}$ folgt unmittelbar
		\begin{align*}
			|Q_{[0,\infty)}(f) - Q_{h,T}(f)| &\le |Q_{[0,\infty)}(f) - Q_{[0,T]}(f)| + |Q_{[0,T]}(f) - Q_{h,T}(f)| \\
			&= |Q_{(T,\infty)}(f)| + |Q_{[0,T)}(f) - Q_{h,T}(f)| \\
			&\le \norm{f}_{\infty}\int_{T}^{\infty}{e^{-x} dx} + \left|-\frac{T}{12}h^{2}f^{''}(\xi)\right| \\
			&= C_1e^{-T} + T\frac{h^{2}}{12}|f^{''}(\xi)| \\
			& \le C_1\varepsilon_T + \norm{f^{''}}_{\infty}T\frac{h^{2}}{12} \\
			& = C_1\varepsilon_T + C_2T\varepsilon_h 
		\end{align*}
	\end{myproof}

		\begin{bemerkung}
			Damit der Fehler minimal ist, soll $\varepsilon_T \approx T\varepsilon_h$ gelten. Anders angeschrieben soll also $h \approx \sqrt{\frac{12e^{-T}}{T}}$ bzw. $n \approx \sqrt{\frac{T^{3}}{12e^{-T}}}$ erfüllt sein.
		\end{bemerkung}
	Die Summierte Trapezregel kann zum Beispiel wie folgt implementiert werden:
	\lstinputlisting[language=c,  caption=Summierte Trapezregel]{../code/aufgabe-b/c/summierte-trapezregel.cpp}
	
	
		%\newpage
		
		%\begin{figure}[H]
		%	\begin{center}
		%		\includegraphics[width=0.8\textwidth]{parameterth.png}
		%	\end{center}
		%	\caption{Fehler abhängig von n.}
		%	\label{fig:parameterth}	
		%\end{figure}
		 
		\subsection{Von T abh\"angiger Fehler}
	Da f\"ur Funktionen $f,g$, die auf dem Intervall $[a,b]$ Riemann-integrierbar sind gilt \autocite[vgl.][271]{ana2}: 
	\begin{align*}
		f(x)\leq g(x): x \in [a,b]\Rightarrow \int_{a}^{b}f(x)dx \leq \int_{a}^{b}g(x)dx,
	\end{align*}
	ergibt sich f\"ur den Fehler $C_1\varepsilon_T$ aus \cref{eq:trapez-fehler} f\"ur andere Gewichtsfunktionen $\omega : f(x)\omega(x) \leq f(x)e^{-x}, x \in [T,\infty)$
	\begin{align*}
		\abs{\int_{0}^{\infty}f(x)e^{-x}dx-\int_{0}^{T}f(x)e^{-x}dx} = \abs{\int_{T}^{\infty}f(x)e^{-x}dx} \geq \\
		\abs{\int_{T}^{\infty}f(x)\omega(x)dx} =  \abs{\int_{0}^{\infty}f(x)\omega(x)dx-\int_{0}^{T}f(x)\omega(x)dx}
	\end{align*}	
	Der bei \cref{trapez} durch das Abschneiden des Intervalls $[0,\infty)$ entstehende Fehler ist also stark von der Wahl der Gewichtsfunktion abh\"angig.(\cref{fig:f-vergleich-int}) 
	Durch die Wahl einer geeigneten Gewichtsfunktion kann die Konvergenz des Integrationsverfahrens also beschleunigt werden.
	
	
	\begin{figure}[H]
		\begin{subfigure}[t]{0.5\textwidth}
			
			\includegraphics[width=\linewidth]{../plots/aufgabe-e-vergleich.png}
			\subcaption{$f(x)\omega(x)$ f\"ur verschiedene $\omega(x)$}\label{fig:f-vergleich}
			
		\end{subfigure}
		\begin{subfigure}[t]{0.5\textwidth}
			\includegraphics[width=\linewidth]{../plots/aufgabe-e-vergleich-integral.png}
			\subcaption{Von T abh\"angiger Fehler f\"ur verschiedene Gewichtsfunktionen}\label{fig:f-vergleich-int}
		\end{subfigure}
		\caption{Vergleich verschiedener Gewichtsfunktionen}
	\end{figure}

	F\"ur eine Zerlegung $Z = \{\xi_j:j=0,\dots,n(Z):\xi_j = jh+\xi_0\} , h\in \mathbb{R}$ von $[0,T]$ bleibt f\"ur $T\to\infty$ nur der von $h$ abh\"angige Fehler.
	\lstinputlisting[language=Python, firstline = 4, lastline = 20, caption=Implementierung des Fehlers mit fixem h]{fehlerhsinx.py} 	
	\begin{figure}[H]
		\begin{subfigure}[t]{0.5\textwidth}
			\includegraphics[width=\linewidth]{Fehlerplothsinx.png}
			\subcaption{$f(x) := \frac{sin(x)}{x}e^{-x}$} \label{fig:fehlerplothsinx}
		\end{subfigure}
		\begin{subfigure}[t]{0.5\textwidth}
			\includegraphics[width=\linewidth]{Fehlerplothexp.png}
			\subcaption{$f(x) := \frac{1}{e^{x}+7}e^{-x}$} \label{fig:fehlerplothexp}
		\end{subfigure}
		\caption{Fehler abhängig von T.}
		\label{fig:fehlerploth}
	\end{figure}

	Man bemerkt in \cref{fig:fehlerploth}, dass bei verschieden Funktionen sich die Konvergenzgeschwindigkeit des Integrationsfehler ändert. 
	
	\subsection{Von n abh\"angiger Fehler}
	\begin{mytheorem}\label{satz:fehler-kruemmung}
		F\"ur auf einem Intervall $[a,b]:0\leq a< b$ Riemann-integrierbare Funktionen $f,g\in C^2([a,b])$ und eine Zerlegung $Z = \{\xi_{i}:i=0,\dots,n(Z)\}$ von $[a,b]$ ist der von n abh\"angige Fehler f\"ur  $f$ durch
		\begin{align*}
		\tau:(C^2([a,b]),M)\to\mathbb{R},\tau(f,Z) = \left|\sum_{i=1}^{n(Z)}\left(\int_{\xi_{i-1}}^{\xi_i}f(x)dx - \left(d(\xi_{i-1},\xi_i)\frac{f(\xi_{i-1})+f(\xi_i)}{2}\right)\right)\right|
		\end{align*}
		gegeben, wobei $d(\xi_{i-1},\xi_{i})$ der Abstand zwischen $\xi_{i-1}$ und $\xi_{i}$ ist und $M$ die Menge der Zerlegungen von $[a,b]$ ist und es gilt:
		\begin{align*}
		\forall x\in [a,b]:|\frac{d^2}{dx^2}f(x)|\geq |\frac{d^2}{dx^2}g(x)|\geq 0\Rightarrow \tau(f,Z)\geq\tau(g,Z).
		\end{align*}
	\end{mytheorem}
	\begin{myproof}
		Dass $\tau(f,Z)$ der Fehler f\"ur eine der in \cref{satz:fehler-kruemmung} beschriebenen Funktionen $f$ und eine Zerlegung $Z$ ist, folgt direkt aus der Definition der summierten Trapezformel. \\
		Da $\forall x\in [a,b]: |\frac{d^2}{dx^2}f(x)|\geq |\frac{d^2}{dx^2}g(x)|$, gen\"ugt es ein einzelnes Intervall $(\xi_{i-1},\xi_i)$ zu betrachten. F\"ur $c\in\mathbb{R}$ gilt 
		\begin{gather*}
		\tau(f,\{\xi_{i-1},\xi_i\}) = \left|\int_{\xi_{i-1}}^{\xi_i}f(x)dx - \left(d(\xi_{i-1},\xi_i)\frac{f(\xi_{i-1})+f(\xi_i)}{2}\right)\right| =\\
		\left|\int_{\xi_{i-1}}^{\xi_i}f(x)dx - \int_{\xi_{i-1}}^{\xi_i}cdx + d(\xi_{i-1},\xi_i)c - \left(d(\xi_{i-1},\xi_i)\frac{f(\xi_{i-1})+f(\xi_i)}{2}\right)\right| = \\
		\left|\int_{\xi_{i-1}}^{\xi_i}f(x) - cdx  - \left(d(\xi_{i-1},\xi_i)\frac{f(\xi_{i-1})-c+f(\xi_i)-c}{2}\right)\right| = \tau(f-c,\{\xi_{i-1},\xi_i\}).
		\end{gather*}
		Es kann also $min(|f(\xi_{i-1})|,|f(\xi_{i})|) = 0$ vorausgesetzt werden. F\"ur die Verbindungsgerade $s(x)\coloneqq f(\xi_{i-1}) + \frac{f(\xi_{i}) - f(\xi_{i-1})}{\xi_{i} - \xi_{i-1}}x$ der Punkte $(\xi_{i-1},f(\xi_{i-1}))$ und $(\xi_{i},f(\xi_{i}))$ gilt somit $min(|s(\xi_{i-1})|, |s(\xi_{i})|) = 0$ und:
		\begin{gather*}
		\tau(f,\{\xi_{i-1},\xi_i\}) =\left|\int_{\xi_{i-1}}^{\xi_i}f(x)dx - \left(d(\xi_{i-1},\xi_i)\frac{f(\xi_{i-1})+f(\xi_i)}{2}\right)\right| =\\
		\left|\int_{\xi_{i-1}}^{\xi_i}f(x)dx - \int_{\xi_{i-1}}^{\xi_i}s(x)dx + d(\xi_{i-1},\xi_i)\frac{s(\xi_{i-1})+s(\xi_i)}{2} - \left(d(\xi_{i-1},\xi_i)\frac{f(\xi_{i-1})+f(\xi_i)}{2}\right)\right| = \\
		\left|\int_{\xi_{i-1}}^{\xi_i}f(x) - s(x)dx  - \left(d(\xi_{i-1},\xi_i)\frac{f(\xi_{i-1})-s(\xi_{i-1})+f(\xi_i)-s(\xi_i)}{2}\right)\right| = \tau(f-s,\{\xi_{i-1},\xi_i\})
		\end{gather*} 
		Es existiert also f\"ur jedes $f$  ein  $\hat{f}$ der Form $\hat{f}(x)=(f(x) - s(x) - c)$, das wie oben gezeigt $\hat{f}(\xi_{i-1}) = \hat{f}(\xi_{i}) = 0$, $\tau(f,Z) = \tau(\hat{f},Z)$ und somit
		\begin{align*}
			\tau(f,\{\xi_{i-1},\xi_i\}) = \left|\int_{\xi_{i-1}}^{\xi_i}\hat{f}(x)dx - \left(d(\xi_{i-1},\xi_i)\frac{\hat{f}(\xi_{i-1})+\hat{f}(\xi_i)}{2}\right)\right| = \left|\int_{\xi_{i-1}}^{\xi_i}\hat{f}(x)dx\right|
		\end{align*}
		erf\"ullt. Weiters gilt
		\begin{align*}
		\frac{d^2}{dx^2}\hat{f}(x) = \frac{d^2}{dx^2} (f(x) - s(x) - c) = \frac{d^2}{dx^2}f(x) - \frac{d^2}{dx^2}s(x) - \frac{d^2}{dx^2}c= \frac{d^2}{dx^2}f(x)
		\end{align*}
		und weil au\ss erdem $f,g\in C^2([a,b])$ und $\forall x \in [a,b]:|\frac{d^2}{dx^2}f(x)| \geq |\frac{d^2}{dx^2}g(x)| \geq 0$, kann f\"ur den Fehler wegen $\left|\int_{\xi_{i-1}}^{\xi_i}\hat{f}(x)dx\right| = \left|\int_{\xi_{i-1}}^{\xi_i}-\hat{f}(x)dx\right|$ vorausgesetzt werden, dass $\forall x\in [a,b]:\frac{d^2}{dx^2}f(x) \leq \frac{d^2}{dx^2}g(x) \leq 0$ gilt.\\
		Nun folgt aus dem Satz von Rolle die Existenz von $x_{1,2}$ f\"ur die gilt, dass $\hat{f}'(x_1) = \hat{g}'(x_2) = 0$ ist. Weil $\hat{f}'$ und $\hat{g}'$ stetig und streng monoton sind und $[a,b]$ kompakt ist, ist $\hat{f}'([a,b])\cap \hat{g}'([a,b]) = [u,v]$ ein nicht leeres, abgeschlossenes Intervall. \\
		Sei nun $h(x)\coloneqq f(x)-g(x),h'(x)= f'(x)-g'(x)$, dann folgt aus der Surjektivit\"at von $\hat{f}',\hat{g}'$ auf $[u,v]$ die Existenz von $\zeta_1,\zeta_2$, f\"ur die folgendes gilt:
		\begin{align*}
			\hat{f}'(\zeta_1) = u \land \hat{g}'(\zeta_1) \geq u\\
			\hat{f}'(\zeta_2) = v \land \hat{g}'(\zeta_2) \leq v
		\end{align*}
		Es ist also entweder $h'(\zeta_1) \leq 0\leq h'(\zeta_2)$ oder $h'(\zeta_1) \geq 0\geq h'(\zeta_2)$. Aus dem Zwischenwertsatz folgt die Existenz eines $\zeta\in [u,v]$ f\"ur das $h'(\zeta) =0$ ist, woraus $\hat{f}'(\zeta) = \hat{g}'(\zeta)$ folgt. 
		
		Das Intervall $[a,b]$ kann nun in zwei Intervalle $[a,\zeta],[\zeta,b]$ aufgeteilt werden. F\"ur $h''(x) = f''(x)-g''(x)$ folgt dann mit dem Mittelwertsatz f\"ur $\tilde{a}\in [a,\zeta]$ und $\tilde{b}\in [\zeta,b]$:
		\begin{gather*}
			0\geq h''(x_0) = \frac{\hat{f}'(\zeta)-\hat{g}'(\zeta)-\hat{f}'(\tilde{a})+\hat{g}'(\tilde{a})}{\zeta-\tilde{a}} = \frac{\hat{g}'(\tilde{a})-\hat{f}'(\tilde{a})}{\zeta-\tilde{a}}\\
			\hat{f}'(\tilde{a})\leq \hat{g}'(\tilde{a})\\
			0\geq h''(x_0) = \frac{\hat{f}'(\tilde{b})-\hat{g}'(\tilde{b})-\hat{f}'(\zeta)+\hat{g}'(\zeta)}{\tilde{b}-\zeta} = \frac{\hat{f}'(\tilde{b})-\hat{g}'(\tilde{b})}{\zeta-\tilde{b}}\\
			\hat{g}'(\tilde{b})\geq \hat{f}'(\tilde{b})
		\end{gather*}
		Durch erneutes anwenden des Mittelwertsatzes folgt nun:
		\begin{gather*}
		0\leq h'(x_0) = \frac{\hat{f}(\tilde{a})-\hat{g}(\tilde{a})-\hat{f}(a)+\hat{g}(a)}{\tilde{a}-a} = \frac{\hat{g}(\tilde{a})-\hat{f}(\tilde{a})}{\tilde{a}-a}\\
		\hat{f}(\tilde{a})\geq \hat{g}(\tilde{a})\\
		0\geq h'(x_0) = \frac{\hat{f}(b)-\hat{g}(b)-\hat{f}(\tilde{b})+\hat{g}(\tilde{b})}{b-\tilde{b}} = 
		\frac{\hat{g}(\tilde{b})-\hat{f}(\tilde{b})}{b-\tilde{b}}\\
		\hat{f}(\tilde{b})\geq \hat{g}(\tilde{b})
		\end{gather*}
		und somit $\forall x\in[a,b]:\hat{f}(x)\geq\hat{g}(x)$. Es gilt also  
		\begin{align*}
			\tau(f,\{\xi_{i-1},\xi_i\}) = \int_{\xi_{i-1}}^{\xi_{i}}\hat{f}(x)dx\geq \int_{\xi_{i-1}}^{\xi_{i}}\hat{g}(x)dx = \tau(g,\{\xi_{i-1},\xi_i\})\\
		\end{align*}
		und 
		\begin{gather*}
		\tau(f,Z)\geq\tau(g,Z).
		\end{gather*}
		
	\end{myproof}

	
	\begin{figure}[H]
		\centering
		\begin{subfigure}[c]{0.45\textwidth}
			\includegraphics[width=\linewidth]{../plots/fehler-kruemmung.png}
			\subcaption{$f,h:|f''|<|h''|$} \label{fig:fehler-kruemmung-normal}
		\end{subfigure}
	\hfill
		\begin{subfigure}[c]{0.45\textwidth}
			\includegraphics[width=\linewidth]{../plots/fehler-kruemmung-satz.png}
			\subcaption{Laut \cref{satz:fehler-kruemmung} existierende $\hat{f},\hat{h}$} \label{fig:fehler-kruemmung-satz}
		\end{subfigure}
		\caption{Veranschaulichung von \cref{satz:fehler-kruemmung}}
		\label{fig:fehler-kruemmung}
	\end{figure}
	
	
	Wird nun $T$ fix gehalten und $h$ variiert, bleibt f\"ur $h\to 0$ der von $T$ abh\"angige Fehler \"uber. 
	\lstinputlisting[language=Python, firstline = 4, lastline = 20, caption=Implementierung des Fehlers mit fixem T]{fehlerTsinx.py}
	\begin{figure}[H]
		\begin{subfigure}[t]{0.5\textwidth}
			\includegraphics[width=\linewidth]{FehlerplotTsinx.png}
			\subcaption{$f(x) := \frac{sin(x)}{x}e^{-x}$} \label{fig:fehlerplottsinx}
		\end{subfigure}
		\begin{subfigure}[t]{0.5\textwidth}
			\includegraphics[width=\linewidth]{FehlerplotTexp.png}
			\subcaption{$f(x) := \frac{1}{e^{x}+7}e^{-x}$} \label{fig:fehlerplottexp}
		\end{subfigure}
		\caption{Fehlerplot abhängig von h.}
		\label{fig:fehlerplott}
	\end{figure}
	Man sieht nun aus \cref{fig:fehlerplott}, wobei die gerade Funktion als Vergleich dient wie sich der Fehler verhalten sollte, dass bei kleiner werdenden h sich der Fehler ab einem gewissen Wert nicht mehr ändert. D.h. man ist zum von T abhängigen Fehler gelangt.   
	
	%sebastians teil
	\newpage
	\section{Gau\ss-Laguerre Quadratur}\label{gauss}
	Zun\"achst  sei bemerkt, dass\autocite[45]{skript}:
	\begin{mykorollar}\label{4.15}
		Es existiert eine eindeutig bestimmte Folge $(q_n)$ von Polynomen der Form
		\begin{align*}
		q_0(x)&\coloneqq 1\\
			q_n(x)&\coloneqq x^n+r_{n-1}(x),\quad x\in [a,b]
		\end{align*}
		mit $r_{n-1}\in \prod_{n-1}$, die die Orthogonalith\"atsbeziehungen  $(q_n,q_m)_w=0,\; n\neq m $ erf\"ullen.
	\end{mykorollar}

	\noindent und
	\begin{mylemma}
		Jedes der orthogonalen Polynome $q_n\in \prod_n$, aus \cref{4.15} besitzt $n$ einfache Nullstellen in $(a,b)$.
	\end{mylemma}

	\noindent Woraus sich dann folgender Satz ergibt:
	\begin{mytheorem}\label{aufgabec-satz}
		%Korollar 4.15  \autocite[45]{skript}
		Die Orthogonalpolynome $q_n \in \prod_n$ aus \cref{4.15} sind eindeutig durch die folgenden 3 Terme gegeben
		\begin{align*}
			q_0(x) &\coloneqq 1\\
			q_1(x) &\coloneqq (x-\beta_0) q_0 = x-\beta_0 \\
			 q_{n+1}(x) &\coloneqq (x-\beta_n)q_n(x)-\gamma^2_n q_{n-1}(x) \quad \text{f\"ur}  \quad n\geq 1
		\end{align*}
		mit
		\begin{align*}
			\beta_n\coloneqq \frac{(xq_n,q_n)_w}{\norm{q_n}^2_w} \quad \text{und} \quad 
			\gamma_n\coloneqq \frac{\norm{q_n}_w}{\norm{q_{n-1}}_w}
		\end{align*}
		Weiter sind die Eigenwerte der Matrix
		\begin{align*}
		T\coloneqq 
			\begin{pmatrix}
			\beta_0 		 & 	\gamma_1 	&  						& &\\
			\gamma_1 	 & \beta_1 			& \gamma_2	  & & \\
			 				    	&  \gamma_2   & \ddots 		      & \ddots &\\
			 				    	&					& \ddots &	\beta_{n-1} & \gamma_n\\
			 				    	&	&	&\gamma_n &\beta_n&\\
			\end{pmatrix}
			\in \mathbb{R}^{(n+1)\times (n+1)}
		\end{align*}
		die Nullstellen $x_0,\dots ,x_n$ des $(n+1)$-ten Orthogonalpolynoms $q_{n+1}$ und die zugeh\"origen Gewichte der Gau\ss-Quadratur sind gegeben durch
		\begin{align*}
			\alpha_j \coloneqq \left(\frac{(v_j)_1}{\norm{v_j}_2}\right)^2 \int_{a}^{b} w(x)dx,\quad j=0,\dots,n
		\end{align*}
		wobei $(v_j)_1$ die 1.Komponente eines Eigenvektors $v_j \in \mathbb{R}^{n+1}\setminus \{0\}$ zum Eigenwert $x_j$ ist.
	\end{mytheorem}
	\begin{myproof}
				Durch die Gewichtsfunktion $w(x) = exp(-x)$ haben die Polynome und die Matrix folgende Form:	
		\begin{align*}	
		p_0(x) &\coloneqq 1\\	
		p_1(x) &\coloneqq (x-1) p_0 = x-1 \\	
		p_{n+1}(x) &\coloneqq (x-2n-1)p_n(x)-n^2 p_{n-1}(x) \quad \text{f\"ur}  \quad n\geq 1	
		\end{align*} 	
		und 	
		\begin{align*}	
		T_n\coloneqq 	
		\begin{pmatrix}	
		1 		     & 	-1			 	&  				  & &\\	
		-1		 	 & 3				& 			   	  & & \\	
		&   		 & \ddots 		    & \ddots 			&\\	
		&					& \ddots 	&					& -n+1\\	
		&		&			&-n+1 		&2n-1				&\\	
		\end{pmatrix}	
		\end{align*}	
		Mit vollständiger Induktion wird gezeigt, dass $T_nv_{k,n} = x_kv_{k,n}$ gilt, mit dem Eigenvektor 	
		$v_{k,n} := (\tau_0,\tau_1p_1(x_k),\ldots,\tau_{n-1}p_{n-1}(x_k))^{T}$ und $\tau_j = \frac{(-1)^{j}}{j!}$ zum Eigenwert $x_k$.	
		\begin{itemize}	
			\item[I.V.] 	
			Es gilt $T_{n-1}v_{k,n-1} = x_kv_{k,n-1}$	
			\item[I.A.] 	
			Mit n = 1 folgt unmittelbar	
			\begin{align*}	
			T_1v_{k,1} = (1)(\tau_0) = \tau_0 \stackrel{!}{=} x_k\tau_0 \Leftrightarrow T_1v_{k,1} - x_k\tau_0 = \tau_0 - x_k\tau_0 = 1 - x_k = 0	
			\end{align*}	
			Also ist der Eigenwert $x_k = 1$ was bei genauerer Betrachtung auch die Nullstelle des Orthogonalpolynoms $p_1(x_k) = x_k - 1 = 0$ ist. 	
			\item[I.S.] $n \rightarrow n + 1$	
			\begin{align*}	 	
				T_nv_{k,n}
				& =
				\begin{pmatrix}	
					1 		     & 	-1			 	&  				  & &\\	
					-1		 	 & 3				& 			   	  & & \\	
					&   		 & \ddots 		    & \ddots 			&\\	
					&					& \ddots 	&					& -n+1\\	
					&		&			&-n+1 		&2n-1				&\\	
				\end{pmatrix}
				\begin{pmatrix}
					\tau_0 \\ \tau_1p_1(x_k) \\ \vdots \\ \tau_{n-1}p_{n-1}(x_k)
				\end{pmatrix} \\
				& = 
				\begin{pmatrix}
				\tau_0 -\tau_1p_1(x_k)  \\ -\tau_0 + 3\tau_1p_1(x_k) - 2\tau_2p_1(x_k)\\ \vdots \\ (-n+1)\tau_{n-2}p_{n-2}(x_k) + (2n-1)\tau_{n-1}p_{n-1}(x_k) 
				\end{pmatrix} \\
				& \stackrel{!}{=}
				x_k	
				\begin{pmatrix}
				\tau_0 \\ \tau_1p_1(x_k) \\ \vdots \\ \tau_{n-1}p_{n-1}(x_k)
				\end{pmatrix}
			\end{align*}
			Durch einsetzen der Rekursionsformel für die Orthogonalpolynome $p_n$ auf der linken Seite für die ersten $n-1$ Einträge, folgt direkt die rechte Seite. Für den letzten Eintrag gilt nun
			\begin{align*}
				&x_k\tau_{n-1}p_{n-1}(x_k) = (-n+1)\tau_{n-2}p_{n-2}(x_k) + (2n-1)\tau_{n-1}p_{n-1}(x_k) \\
				\Leftrightarrow &  x_k\tau_{n-1}p_{n-1}(x_k) - (-n+1)\tau_{n-2}p_{n-2}(x_k) - (2n-1)\tau_{n-1}p_{n-1}(x_k) = \\
				& \tau_{n-1}p_{n-1}(x_k)(x_k-2n+1) + (n-1)\tau_{n-2}p_{n-2}(x_k) = 0
			\end{align*} 
			Sei nun o.B.d.A $n$ ungerade und mit Erweiterung des rechten Terms und einsetzen von $\tau_j$ folgt
			\begin{align*}
				&\frac{p_{n-1}(x_k)(x_k-2n+1)}{(n-1)!} - \frac{(n-1)^2p_{n-2}(x_k)}{(n-1)!} = 0 \\
				\Leftrightarrow & p_{n-1}(x_k)(x_k-2(n-1)-1) - (n-1)^2p_{n-2}(x_k) = p_n(x_k) = 0
			\end{align*}
			$x_k$ sind also die Nullstellen des Polynoms $p_n$ und sind zugleich die Eigenwerte von $T_n$, wegen $p_n(x) = det(T_n(x))$. \autocite[vgl. Bsp. 40]{ubung}
		\end{itemize}
		Was noch zu  zeigen bleibt, ist dass die Gewichte folgende Form haben
		\begin{align*}
			\alpha_j &= \left(\frac{(v_j)_1}{\norm{v_j}_2}\right)^2 \int_{a}^{b} w(x)dx = \frac{((v_j)_1)^2}{(\norm{v_j}_2)^2} \int_{0}^{\infty} e^{-x}dx \\
			& = \frac{1}{\norm{v_j}_2^2}
		\end{align*}
		Bei Betrachtung des Ausdrucks 
		\begin{align*}
			\sum_{j=0}^{n-1} \sum_{l=0}^{j+1} w_{l,n} \tau_j^2 p_j(x_k)p_j(x_l)
		\end{align*}
		folgt nun
		\begin{align*}
			\sum_{j=0}^{n-1} \sum_{l=0}^{j+1} w_{l,n} \tau_j^2 p_j(x_k)p_j(x_l) &= \sum_{j=0}^{n-1} \tau_j^2 p_j(x_k) \int_{0}^{\infty} p_j(x)e^{-x}dx \\
			& = \tau_0^2 p_0(x_k) \int_{0}^{\infty} p_0(x)e^{-x}dx + \sum_{j=1}^{n-1} \tau_j^2 p_j(x_k) \int_{0}^{\infty} p_j(x)e^{-x}dx \\
			& = 1 + \sum_{j=1}^{n-1} \tau_j^2 p_j(x_k) \cdot 0 \\
			& = 1
		\end{align*}
		und andererseits
		\begin{align*}
			\sum_{j=0}^{n-1} \sum_{l=0}^{j+1} w_{l,n} \tau_j^2 p_j(x_k)p_j(x_l) 
			 &= \sum_{j=0}^{n-1} \sum_{l=0}^{j+1} w_{l,n} \tau_j^2 p_j(x_k)^2 \delta_{l,k}= \\
			 \sum_{j=0}^{n-1} w_{k,n} \tau_j^2 p_j(x_k)^2 
			 &= w_{k,n} \sum_{j=0}^{n-1}  \tau_j^2 p_j(x_k)^2 
			 = w_{k,n}\norm{v_{k,n}}_2^2	
		\end{align*}
		Also
		\begin{align*}
			w_{k,n}\norm{v_{k,n}}_2^2 = 1 \Rightarrow  w_{k,n} = \frac{1}{\norm{v_{k,n}}_2^2}
		\end{align*}
	\end{myproof}
	\newpage
	Somit ist also die in \Cref{aufgabed-code-main,aufgabed-code-gauss} zu sehende Vorgehensweise gerechtfertigt.
	\lstinputlisting[language=c,firstline=17,lastline=21,label={aufgabed-code-main},caption={M\"ogliche Implementierung von \cref{aufgabec-satz} - main.cpp}]{../code/aufgabe-d/main.cpp}
	\lstinputlisting[language=c,firstline=14,lastline=50,label={aufgabed-code-gauss},caption={gauss.cpp}]{../code/aufgabe-d/gauss.cpp}
	
	\begin{figure}[H]
		\begin{center}
			\includegraphics[width=0.8\textwidth]{../plots/quadraturpunkte-n-zusammen.png}
		\end{center}
		\caption{Quadraturpunkte und zugeh\"orige Gewichte}
		\label{fig:quadraturpunkte}
	\end{figure}

	\begin{bemerkung}
		Um, wie in \cref{fig:f-vergleich-int} auch Integrale der Form $\int_{a}^{\infty}f(x)\omega(x)dx:a>0$ zu berechnen kann f\"ur Funktionen $f,\omega:\int_{0}^{\infty}f(x)\omega(x)dx<\infty$ wegen \autocite[vgl.][278]{ana2}
		\begin{align*}
		\int_{a}^{\infty}f(x)\omega(x)dx = \int_{0}^{\infty}f(x+a)\omega(x+a)dx = \int_{0}^{\infty}f(x+a)\omega(x+a)\omega^{-1}(x)\omega(x)dx
		\end{align*}
		eine neue Funktion $g$ mit $g(x) \coloneqq f(x+a)\omega(x+a)\omega^{-1}(x)$ definiert werden f\"ur die gilt:
		\begin{align*}
		\int_{a}^{\infty}f(x)\omega(x)dx = \int_{0}^{\infty}g(x)\omega(x)dx \approx  \sum_{i=1}^{n}w_ig(x_i).
		\end{align*}
		Weiters k\"onnen wegen $\int_{a}^{b}f(x)\omega(x)dx = \int_{a}^{c}f(x)\omega(x)dx - \int_{b}^{c}f(x)\omega(x)dx$ f\"ur $0\leq a\leq b\leq c$ auch Integrale der Form $\int_{a}^{b}f(x)\omega(x)dx$ berechnet werden.
	\end{bemerkung}
	
	\newpage
	\section{Vergleich der Methoden aus Abschnitt 2 und 3} %cref geht bei lmark in der kopfzeile nicht
	
	%Bei der summierten Trapezformel wurde für jedes n ein optimales T gewählt wie in \cref{fig:parameterth}
	%F\"ur die Funktionen $f:[0,\infty)\to \mathbb{R},f(x)=\frac{1}{e^x+7}$ und $\omega:[0,\infty)\mapsto \mathbb{R},\omega(x)=e^{-x}$, konvergiert die Gauss-Laguerre Quadratur wesentlich schneller gegen $\int_{0}^{\infty}f(x)\omega(x)dx$ als die in \cref{trapez} beschriebene Integrationsmethode.(\cref{fig:vergleich})
	Wie bereits gezeigt, ist das Konvergenzverhalten der Summierten Trapezregel stark von der zu integrierenden Funktion abh\"angig. Um die Konvergenz zu beschleunigen kann entweder eine st\"arker fallende Gewichtsfunktion gew\"ahlt werden, um den von $T$ abh\"angigen Fehler zu verkleinern oder eine schw\"acher gekr\"ummte Funktion, um den von $n$ abh\"angigen Fehler zu minimieren.  
		\begin{figure}[H]
		\begin{center}
			\includegraphics[width=0.8\textwidth]{../plots/aufgabe-e-vergleich-trapez-gewichtsfnkt.png}
		\end{center}
		\caption{Fehler der Summierten Trapezregel f\"ur unterschiedlich stark fallende Gewichtsfunktionen und $f(x)=\frac{1}{e^x+7}$}
		\label{fig:gewichtsfunktionen-vergleich}	
	\end{figure}
	

	Wie in \cref{fig:gewichtsfunktionen-vergleich} zu sehen ist, geht der Fehler f\"ur $e^{-x^2}$ wesentlich schneller gegen Null als bei $e^{-x}$. Trotzdem bleibt die Konvergenz langsamer als bei der in \cref{gauss} beschriebenen Gau\ss -Quadratur (\cref{fig:gauss-vs-trapez}).
	
	
	\begin{figure}[H]
		\begin{subfigure}[t]{0.5\textwidth}
			
			\includegraphics[width=\linewidth]{../plots/vergleich-gauss-trapez-expx.png}
			\subcaption{$\int_{0}^{\infty}\frac{e^{-x}}{e^x+7}dx$}\label{fig:gauss-vs-trapez-expx}
			
		\end{subfigure}
		\begin{subfigure}[t]{0.5\textwidth}
			\includegraphics[width=\linewidth]{../plots/vergleich-gauss-trapez-expxx.png}
			\subcaption{$\int_{0}^{\infty}\frac{e^{-x^2}}{e^x+7}dx$}\label{fig:gauss-vs-trapez-expxx}
		\end{subfigure}
	\caption{Vergleich der Integrationsverfahren aus \cref{trapez,gauss} f\"ur verschiedene Gewichtsfunktionen}\label{fig:gauss-vs-trapez}
	\end{figure}

	%\begin{figure}[H]
	%	\begin{center}
	%		\includegraphics[width=0.8\textwidth]{Fehlerhvergleich.png}
	%	\end{center}
	%	\caption{Fehler abhängig von h bei verschiedenen Gewichtsfunktionen}
	%	\label{fig:fehlerhverhleich}	
	%\end{figure}
	%abbildung trapez e^-x vs e^-x^2
	
	%und abbildung trapez e^-x^2 vs gauss
	

	%\section{Gauss–Hermite Quadratur?}
	 % Um auch uneigentliche Integrale von $-\infty$ bis $\infty$ zu berechnen, kann als Gewichtsfunktion $e^{-x^2}$ verwendet werden. Es gilt:
	 % \begin{align*}
	%  	\int_{-\infty}^{\infty}e^{-x^2}f(x)dx \approx \sum_{i=1}^{n}w_if(x_i)
	 % \end{align*}
	%literatur,abbildungsverzeichnis etc.
	\newpage
	\printbibliography
	\listoffigures
	\thispagestyle{firststyle}
	
\end{document}

